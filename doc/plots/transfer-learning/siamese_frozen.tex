\documentclass{article}

\usepackage{tikz}
\usetikzlibrary{shapes,arrows,shadows}
\usepackage{amsmath,bm,times}
%%%<
\usepackage[active,tightpage]{preview}
\PreviewEnvironment{tikzpicture}
\setlength\PreviewBorder{5pt}%

\usepackage{graphicx}
\usepackage{color}
\usepackage{tikz}
\usepackage{pgfplots}
\usepackage{pgf-umlsd}
\usepackage{ifthen}


\begin{document}



\tikzstyle{block} = [draw, rectangle, minimum height=3em, minimum width=3em]
\tikzstyle{virtual} = [coordinate]

\begin{tikzpicture}[auto, node distance=2cm]

    % Define the macro.
    % 1st argument: Height and width of the layer rectangle slice.
    % 2nd argument: Depth of the layer slice
    % 3rd argument: X Offset --> use it to offset layers from previously drawn layers.
    % 4th argument: Options for filldraw.
    % 5th argument: Text to be placed below this layer.
    % 6th argument: Y Offset --> Use it when an output needs to be fed to multiple layers that are on the same X offset.

    \newcommand{\networkLayer}[6]{
        \def\a{#1} % Used to distinguish input resolution for current layer.
        \def\b{0.02}
        \def\c{#2} % Width of the cube to distinguish number of input channels for current layer.
        \def\t{#3} % X offset for current layer.
        \ifthenelse {\equal{#6} {}} {\def\y{0}} {\def\y{#6}} % Y offset for current layer.

        % Draw the layer body.
        \draw[line width=0.25mm](\c+\t,0,0) -- (\c+\t,\a,0) -- (\t,\a,0);                                                      % back plane
        \draw[line width=0.25mm](\t,0,\a) -- (\c+\t,0,\a) node[midway,below] {#5} -- (\c+\t,\a,\a) -- (\t,\a,\a) -- (\t,0,\a); % front plane
        \draw[line width=0.25mm](\c+\t,0,0) -- (\c+\t,0,\a);
        \draw[line width=0.25mm](\c+\t,\a,0) -- (\c+\t,\a,\a);
        \draw[line width=0.25mm](\t,\a,0) -- (\t,\a,\a);

        % Recolor visible surfaces
        \filldraw[#4] (\t+\b,\b,\a) -- (\c+\t-\b,\b,\a) -- (\c+\t-\b,\a-\b,\a) -- (\t+\b,\a-\b,\a) -- (\t+\b,\b,\a); % front plane
        \filldraw[#4] (\t+\b,\a,\a-\b) -- (\c+\t-\b,\a,\a-\b) -- (\c+\t-\b,\a,\b) -- (\t+\b,\a,\b);

        % Colored slice.
        \ifthenelse {\equal{#4} {}}
        {} % Do not draw colored slice if #4 is blank.
        {\filldraw[#4] (\c+\t,\b,\a-\b) -- (\c+\t,\b,\b) -- (\c+\t,\a-\b,\b) -- (\c+\t,\a-\b,\a-\b);} % Else, draw a colored slice.
    }

    %%% Beginning

    % Place nodes
    \node [virtual] at (0,0)                      (input_left)     {};
    %\node [block, right of=input_left] at (0,0)   (model_left)     {Resnet-v2};

    \networkLayer{0.75}{2}{2}{color=white, name=model_left}{Resnet-v2}{}


    %\node [virtual] at (0,2)                       (input_right)     {};
    %\node [block, right of=input_right]   at (0,2) (model_right)     {Resnet-v2};

    \networkLayer{0.75}{2}{2}{color=white, name=model_left}{Resnet-v2}{}

    %\node [block, right of=model_right, rotate=-90]   at (4,1) (loss)     {Contrastive loss};

    \node [virtual] at (8,1)                      (loss_value)     {};

    % Connect nodes
    %\draw [->] (input_left) -- node {$x_B$} (model_left);
    %\draw [->] (input_right) -- node {$x_A$} (model_right);
    %\draw [-] (model_left) -- node {$\theta_1$} (model_right);

    %\draw [->] (model_right) -- node {$\phi(x_A)$} (loss);
    %\draw [->] (model_left) -- node[below] {$\phi(x_B)$} (loss);

    %\draw [->] (loss) -- node[below] {$\mathcal{L}$} (loss_value);


\end{tikzpicture}

\end{document}
